
%%%%%%%%%%%%%%%%%%%%%%%%%%%%%%%%%%%%%%%%%%%%%%%%%%
\chapter{Analysis}
\label{chap:analysis}

To figure out if this is plausible,
we can start with an absolute baseline energy requirement:
four gigajoules per person.
%
No matter how we do it,
whether we use rockets or a cannon or a space elevator,
moving a 65-kilogram person---or 65 kilograms of anything---%
out of the Earth's gravity well requires at least this much energy.

\begin{equation*}
\text{Gravitational potential energy} = \frac{1}{2} \times 65\mathrm{kg} \times \text{(Earth escape velocity)}^{2}
\end{equation*}

The energy required to lift something away from the Earth
is equal to its kinetic energy if it were moving at Earth's escape velocity.

How much is four gigajoules?
It's about a megawatt-hour,
which is what a typical US household consumes in electricity in a month or two.
It's the amount of stored energy in a cargo van full of AA batteries or 90 kg of gasoline.

\begin{figure}[hp]
\centering
\includegraphics[width=\textwidth]{figs/everybody_out_cargo_van.png}
\caption{
    An avalanche of batteries falling out of the open back of a cargo van.
}
\label{fig:battery-van}
\end{figure}

Four gigajoules times seven billion people gives us
$2.8 \times 10^{18}$ joules, or 8 petawatt-hours.
This is about five percent of the world's annual energy consumption.
A lot, but not impossible.

But this is just a minimum.
In practice, everything depends on our means of transportation.
If we're using rockets, it's going to take a lot more.
This is because of a fundamental problem with rockets:
they have to lift their own fuel.

Let's return for a moment to those 90 kilograms of gasoline (about 30 gallons),
because they help illustrate a central problem in space travel.
If we want to launch a 65-kilogram spaceship,
we need to burn around 90 kilograms of fuel.
(Gasoline has an energy per pound comparable to that of rocket fuel,
so we'll stick with that example).
We load that fuel on board---and now our spaceship weighs 155 kilograms.
A 155-kilogram spaceship requires 215 kilograms of fuel,
so we load another 125 kilograms on board\dots

Fortunately, we're saved from an infinite loop---%
where we add 1.3 kilograms for every 1 kilogram we add---%
by the fact that we don't have to carry that fuel all the way up.
We burn it as we go, so we get lighter and lighter,
which means we need less and less fuel.
But we do have to lift the fuel partway.
The formula for how much propellant we need to burn to get moving
at a given speed is given by the Tsiolkovsky Rocket equation~\cite{weisstein}:

\begin{equation*}
\Delta v = v_\text{exhaust} \ln \frac {m_\text{start}} {m_\text{end}}
\end{equation*}

$m_\text{start}$ and $m_\text{end}$
are the total mass of the ship+fuel before and after the burn,
and $v_\text{exhaust}$ is the ``exhaust velocity'' of the fuel,
a number that's between 2.5-4.5 km/s for rocket fuels.

What's important is the ratio between $\Delta v$ and $v_\text{exhaust}$---%
the speed we want to be going compared to the speed that the propellant exits our rocket.
The kilograms of fuel needed per kilogram of ship is $e$ to the power of this number,
which gets big very fast.
For leaving Earth, we need a $\Delta v$ of upwards of 13 km/s,
and $v_\text{exhaust}$ isn't much higher than 4.5 km/s,
which gives a fuel-to-ship ratio of at least $\text{e}^{\frac{13}{4.5}} \approx 20$.

The upshot is that to overcome Earth's gravity using traditional rocket fuels,
a one-ton craft needs 20 to 50 tons of fuel.
Launching all of humanity (total weight: around 400 million tons)
would therefore take tens of trillions of tons of fuel.
That's a lot; if we were using hydrocarbon-based fuels,
it would represent a decent chunk of the world's remaining oil reserves.
And that's not even worrying about the weight of the ship itself, food, water, or our pets
(there are probably around a million tons of pet dog in the US alone).
We'd also need fuel to produce all these ships,
to transport people to the launch sites, and so forth.
It's not necessarily completely impossible,
but it's certainly outside the realm of plausibility.

But rockets aren't our only option.
As crazy as it sounds, we might be better off trying to
%
\begin{enumerate*}[label=(\arabic*)]
\item literally climb into space on a rope
      (see \cref{fig:alternatives:elevator}), or
\item blow ourselves off the planet with nuclear weapons
      (see \cref{fig:alternatives:nuke}).
\end{enumerate*}
%
These are actually serious---if audacious---ideas for launch systems,
both of which have been bouncing around since the start of the Space Age.

\begin{figure}[p]
\centering
  \begin{subfigure}[b]{\linewidth}
  \centering
  \includegraphics{figs/everybody_out_elevator.png}
  \caption{Two figures climbing a rope off the earth.}
  \label{fig:alternatives:elevator}
  \end{subfigure}
  \begin{subfigure}[b]{\linewidth}
  \centering
  \includegraphics{figs/everybody_out_nuke.png}
  \caption{An exploded earth with a craft flying off of it.}
  \label{fig:alternatives:nuke}
  \end{subfigure}
  \vspace*{1em}
\caption{
    Alternative options to using rockets.
}
\label{fig:alternatives}
\end{figure}

The first approach is the ``space elevator'' concept,
a favorite of science fiction authors.
The idea is that we connect a tether to a satellite orbiting
far enough out that the tether is held taut by centrifugal force.
Then we can send climbers up the rope using ordinary electricity and motors,
powered by solar power, nuclear generators, or whatever works best.
%
The biggest engineering hurdle is that the tether
would have to be several times stronger than anything we can currently build.
There are hopes that carbon nanotube-based materials
could provide the required strength---adding this
to the long list of engineering problems
which can be waved away by tacking on the prefix ``nano-''.

The second approach is nuclear pulse propulsion~\cite{schmidt2000nuclear},
a surprisingly plausible method
for getting huge amounts of material moving really fast.
The basic idea is that you toss a nuclear bomb behind you and ride the shockwave.
%
You'd think the spacecraft would be vaporized,
but it turns out that if it has a well-designed shield,
the blast flings away before it has a chance to disintegrate.
If it could be made reliable enough,
this system would in theory be capable of lifting entire city blocks into orbit,
and could---potentially---accomplish our goal.

The engineering principles behind this were solid enough that in the 1960s,
under the guidance of Freeman Dyson,
the US government actually tried to build one of these spaceships.
The story of that effort, dubbed Project Orion,
is detailed in the excellent book of the same name
by Freeman's son George Dyson~\cite{Dyson:2002}.
%
Advocates for nuclear pulse propulsion are still disappointed
that the project was cancelled before any prototypes were built.
Others argue that when you think about what they were trying to do---%
put our entire nuclear arsenal in a box,
hurl it high into the atmosphere, and nuke it repeatedly---%
it's terrifying that it got as far as it did.

So the answer is that while sending one person into space is easy,
getting all of us there would tax our resources
to the limit and possibly destroy the planet.
It's a small step for a man, but a giant leap for mankind.
